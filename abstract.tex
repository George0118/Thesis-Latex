\renewcommand{\abstractname}{Περίληψη}

\begin{abstract}
Σε αυτή τη διατριβή, αντιμετωπίζουμε το πρόβλημα της αντιστοίχισης κόμβων - εξυπηρετητών σε ένα περιβάλλον Ομοσπονδιακής Μάθησης(Federated Learning) πολλαπλών μοντέλων, με στόχο την μεγιστοποίηση της επίδοσής του συγκεντρωτικού/παγκόσμιου μοντέλου. Κάθε ένας από τους εξυπηρετητές εκπαιδεύει ένα ξεχωριστό μοντέλο, ενώ οι κόμβοι διαθέτουν διαφορετική πληροφορία, συνεπώς για κάθε έναν από τους εξυπηρετητές έχουν διαφορετική σημασία. Σε αυτό το πλαίσιο μελετάμε και κατασκευάζουμε αλγορίθμους αντιστοίχισης σε διάφορα σενάρια ώστε να γίνουν εμφανή τα πλεονεκτήματα και μειονεκτήματα καθενός από αυτούς. Προκειμένου να προσδοθεί μια ρεαλιστική εφαρμογή ενός τέτοιου προβλήματος μελετάμε σενάρια εντοπισμού φυσικών κινδύνων (πυρκαγιών, πλημμύρων, σεισμών), μέσω φωτογραφιών, σε ένα περιβάλλον Έξυπνης Πόλης - Δημόσιας Ασφάλειας. Για τους αλγορίθμους αντιστοίχισης καταφεύγουμε στη Θεωρία Παιγνίων (Game Theory), Ενισχυτική Μάθηση (Reinforcement Learning) και στη Μετανοητική Μάθηση (Regret Learning), και στη συνέχεια εκτελούμε τη διαδικασία της Ομοσπονδιακής Μάθησης για να λάβουμε και να συγκρίνουμε τα αποτελέσματά μας.    
  
\begin{keywords}
  Ομοσπονδιακή Μάθηση, Ομοσπονδιακή Μάθηση Πολλαπλών Μοντέλων, Αλγόριθμοι Αντιστοίχισης, Θεωρία Παιγνίων, Ενισχυτική Μάθηση, Μετανοητική Μάθηση, Νευρωνικά Δίκτυα, Συνασπισμοί, Διαμόρφωση Συμμετοχής Κόμβων.
\end{keywords}

\end{abstract}


\begin{abstracteng}
  In this thesis, we address the problem of node-server matching in a multi-model Federated Learning environment, aiming to maximize its performance. Each server trains a separate model, while the nodes possess different information, and therefore, hold varying significance for each server. Within this environment, we study and design matching algorithms under various scenarios to highlight the advantages and disadvantages of each. To provide a realistic application of such a problem, we examine scenarios related to the detection of natural hazards (fires, floods, earthquakes) through photographs in a Smart City - Public Safety environment. For the matching algorithms, we utilize Game Theory, Reinforcement Learning, and Regret Learning and subsequently perform the Federated Learning process to obtain and compare our results.

\begin{keywordseng}
    Federated Learning, Multi-Model Federated Learning, Matching Algorithms, Game Theory, Reinforcement Learning, Regret Learning, Neural Networks, Coalitions, Configuration of Nodes' Participation
\end{keywordseng}
\end{abstracteng}

\begin{acknowledgements}

Θα ήθελα σε αυτό το κομμάτι να ευχαριστήσω κάποια άτομα που με βοήθησαν ιδιαίτερα κατά τη διάρκεια εκτέλεσης της διπλωματικής μου εργασίας, αλλά και καθ' όλη την πορεία μου στο πανεπιστήμιο.

Ευχαριστώ τον Δρ. Συμεών Παπαβασιλείου, καθηγητή στο Εθνικό Μετσόβειο Πολυτεχνείο στον Τομέα Επικοινωνιών, Ηλεκτρονικής και Συστημάτων Πληροφορικής αρχικά για την διδασκαλία του που μου κέντρισε το ενδιαφέρον και στη συνέχεια για την καθοδήγηση και συνεργασία στη διπλωματική μου και τις συμβουλές του για τις μεταπτυχιακές μου σπουδές.

Ευχαριστώ την Δρ. Ειρήνη Ελένη Τσιροπούλου, καθηγήτρια στο Πολιτιακό Πανεπιστήμιο της Αριζόνα για την συνεργασία που είχαμε κατά τη διάρκεια της διπλωματικής μου. Η καθοδήγηση, οργάνωση και υποστήριξη από πλευράς της ήταν ανεκτίμητη και είμαι ευγνώμων που είχα την ευκαιρία να συνεργαστώ μαζί της και με ένα πανεπιστήμιο του εξωτερικού.

Ευχαριστώ την Δρ. Μαρία Διαμαντή, μεταδιδακτορική στο Εθνικό Μετσόβειο Πολυτεχνείο στον Τομέα Επικοινωνιών, Ηλεκτρονικής και Συστημάτων Πληροφορικής για την βοήθειά της στην επίλυση απορειών και προβλημάτων αλλά και στην ομαλή και γρήγορη εξοικείωσή μου με το αντικείμενο και στόχους της διπλωματικής μου εργασίας.

Ευχαριστώ τον Δρ. Γεώργιο Γκούμα, καθηγητή στο Εθνικό Μετσόβειο Πολυτεχνείο στον Τομέα Τεχνολογίας Πληροφορικής και Υπολογιστών για τις συμβουλές του κατά τη διάρκεια των σπουδών μου αλλά και για την μεταδοτική διδασκαλία του.

Ευχαριστώ τον Δρ. Δημήτριο Δελλή για την διαθεσιμότητά του και την άμεση υποστήριξή του για οποιαδήποτε τεχνικά ζητήματα στο υπολογιστικό σύστημα HPC ARIS του GRNET (\href{https://www.hpc.grnet.gr/}{HPC ARIS}).

Ευχαριστώ τον Δρ. Μίνωα Αξενίδη και Δρ Δάκη Παυλίδη για την υποστήριξή τους στην εκτέλεση των πειραμάτων της διπλωματικής μου εργασίας στο υπολογιστικό σύστημα (cluster) Gauss στο Ινστιτούτο Πυρηνικής \& Σωματιδιακής Φυσικής του Εθνικού Κέντρου Έρευνας Φυσικών Επιστημών Δημόκριτος (\href{https://www.demokritos.gr/el/institute/%CE%B9%CE%BD%CF%83%CF%84%CE%B9%CF%84%CE%BF%CF%8D%CF%84%CE%BF-%CF%80%CF%85%CF%81%CE%B7%CE%BD%CE%B9%CE%BA%CE%AE%CF%82-%CF%83%CF%89%CE%BC%CE%B1%CF%84%CE%B9%CE%B4%CE%B9%CE%B1%CE%BA%CE%AE%CF%82-%CF%86/}{INP Demokritos}).

Ευχαριστώ την οικογένειά μου που με υποστήριξε καθ' όλη τη διάρκεια των σπουδών μου, διότι χωρίς αυτούς δεν θα είχα καταφερει να φτάσω έως εδώ.

Ευχαριστώ τους φίλους μου για την υποστήριξή τους, για τις όμορφες στιγμές και αναμνήσεις που θα κρατήσω για πάντα.


\bigskip

\bigskip

\bigskip

\end{acknowledgements}