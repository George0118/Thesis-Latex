\chapter{Συμπεράσματα}

Έπειτα από μελέτη των διαφόρων σεναρίων, αλγορίθμων και αποτελεσμάτων μπορούμε να καταλήξουμε σε κάποια γενικά συμπεράσματα από την διαδικασία αυτής της διπλωματικής εργασίας.

Η Ομοσπονδιακή Μάθηση είναι ένας μηχανισμός που μας επιτρέπει να εξάγουμε πληροφορία από δεδομένα, χωρίς να αποκτούμε συνολική πρόσβαση σε αυτά. Έτσι διασφαλίζουμε την ιδιωτικότητα, που είναι απαραίτητη ιδιαίτερα σε συγκεκριμένους τομείς, πετυχαίνοντας παράλληλα την εκπαίδευση των μοντέλων που χρειαζόμαστε. Στην περίπτωσή μας, είχαμε τρία μοντέλα, τα οποία εκπαιδεύονταν από τους κόμβους του περιβάλλοντός μας. Ο κάθε ένας κόμβος διέθετε μια συγκεκριμένη ποσότητα και τύπο δεδομένων και άρα θα είχε διαφορετική σημασία για κάθε μοντέλο - εξυπηρετητή. Αυτό αποδίδεται στους αλγορίθμους μας με τις συναρτήσεις χρησιμότητας. Τις συναρτήσεις αυτές φαίνεται να διαχειρίζεται καλύτερα ο αλγόριθμος Θεωρίας Παιγνίων, ο οποίος καταφέρνει να κάνει τις καλύτερες αντιστοιχίσεις, κρατώντας τις χρησιμότητες υψηλότερα από τους υπόλοιπους αλγορίθμους. Συνεπώς, είναι ευθύνη μας να χτίσουμε μία αντιπροσωπευτική συνάρτηση χρησιμότητας που να αντικατοπτρίζει ορθά το πρόβλημα και τις παραμέτρους του. 

Όπως είδαμε ακόμη και με μία σχετικά απλή συνάρτηση χρησιμότητας, ο αλγόριθμος Θεωρίας Παιγνίων προσπαθώντας να την μεγιστοποιήσει καταφέρνει να πετύχει υψηλή ροή δεδομένων, χαμηλή ενέργεια μετάδοσης και καλύτερη επίδοση στην Ομοσπονδιακή Μάθηση έναντι των υπόλοιπων αλγορίθμων αντιστοίχισης. Παρ' όλα ταύτα, και οι άλλοι αλγόριθμοι πετυχαίνουν πολύ καλά αποτελέσματα, με τον αλγόριθμο Ενισχυτικής Μάθησης να είναι κοντά ως προς τη χρησιμότητα κόμβων. Είναι επίσης σημαντικό να επαναλάβουμε πως και να γίνει κάποια μη βέλτιστη αντιστοίχιση και να υπάρχει κάποιος μη χρήσιμος κόμβος στον συνασπισμό ενός εξυπηρετητή, ο μηχανισμός της ανάθεσης βαρών για την διαδικασία της Ομοσπονδιακής Μάθησης επιτρέπει να μην επηρεαστεί ιδιαίτερα η επίδοση του συγκεντρωτικού μοντέλου. Μία επιπλέον βελτίωση θα ήταν ένας εξυπηρετητής να μην λαμβάνει υπόψη του κάποιον κόμβο ο οποίος έχει πολύ μικρό βάρος έτσι ώστε να μην καταναλώνονται πόροι για την τοπική εκπαίδευση και μετάδοση των παραμέτρων του μοντέλου. Από την άλλη πλευρά με μη βέλτιστη αντιστοίχιση δεν αξιοποιείται πληροφορία που για κάποιον άλλο εξυπηρετητή θα ήταν χρήσιμη. Έτσι, ξεχωρίζουμε την ικανότητα του αλγορίθμου Θεωρίας Παιγνίων να κάνει βέλτιστη αντιστοίχιση σχεδόν σε κάθε ένα από τα σενάρια που του ανατέθηκαν. Προφανώς ο αλγόριθμος αυτός μπορεί να εφαρμοστεί και σε περιπλοκότερα συστήματα, με περισσότερους κόμβους, εξυπηρετητές και κρίσιμα σημεία, αλλά και με πιο σύνθετες συναρτήσεις χρησιμότητας (όπως είδαμε και στο κεφάλαιο 4).

Στο κεφάλαιο 4, εξετάσαμε μία άλλη προσέγγιση στο υπάρχον πρόβλημα. Οι κόμβοι πέρα από την συσχέτισή τους με τους εξυπηρετητές, μπορούν να διαμορφώσουν την συμμετοχή τους στην διαδικασία της Ομοσπονδιακής Μάθησης. Αυτό δίνει την ευελιξία στο σύστημά μας να μεγιστοποιήσει τις συναρτήσεις χρησιμότητας περαιτέρω, επιτρέποντας σε κόμβους που είναι πιο απομακρυσμένοι ή με λιγότερη πληροφορία να συμμετέχουν λιγότερο. Το μεγάλο πλεονέκτημα των αλγορίθμων Μετανοητικής Μάθησης που εφάρμοσαν την παραπάνω στρατηγική ήταν στην μεγάλη μείωση στην ενέργεια εκπαίδευσης των τοπικών μοντέλων, το οποίο αντίστοιχα μεταφράζεται και σε μείωση του χρόνου που απαιτείται για την εκπαίδευση, αφού οι πιο απομακρυσμένοι κόμβοι τείνουν να συμμετέχουν με λιγότερα δεδομένα, εφ' όσον οι κοντινοί στα κρίσιμα σημεία κόμβοι διαθέτουν την περισσότερη και κύρια πληροφορία. Όπως είδαμε φτάνουμε αρκετά κοντά σε απόδοση στην Ομοσπονδιακή Μάθηση σε σχέση με την απόλυτη προσέγγιση του αλγορίθμου Θεωρίας Παιγνίων. Όμως πετυχαίνουμε καλύτερη χρησιμότητα, με μικρότερες ενέργειες εκπαίδευσης και μετάδοσης και μεγαλύτερη ροή δεδομένων.

Αντίστοιχα, οι δυο διαφορετικές προσεγγίσεις στην Μετανοητική Μάθηση μας δείχνουν πως δεν χρειάζεται να γνωρίζουμε τις ακριβείς συμπεριφορές όλων των κόμβων στο σύστημά μας για να πάρουμε ένα εξίσου καλό, αλλά και πολύ πιο γρήγορο αποτέλεσμα. Ο αλγόριθμος Πλήρους Πληροφορίας τείνει να κάνει καλύτερες επιλογές απ' ότι ο Ελλιπούς Πληροφορίας, αλλά εν' τέλει η διαφορά είναι αρκετά μικρή ώστε να δίνει πλεονέκτημα στον μικρό χρόνο εκτέλεσης του αλγορίθμου Ελλιπούς Πληροφορίας και στην πιο γρήγορη σύγκλισή του. Συνεπώς ο αλγόριθμος αυτός κάνει μια πολύ γρήγορη αντιστοίχιση (παίρνοντας υπόψη το πόσες διαφορετικές επιλογές υπάρχουν για τον κάθε κόμβο) και πετυχαίνει υψηλές χρησιμότητες κατά μέσο όρο μένοντας λίγο πίσω από τον Πλήρους Πληροφορίας. Όσον αφορά την Ομοσπονδιακή Μάθηση έχει την μικρότερη επίδοση, όμως μένει αξιοπρεπώς κοντά στους άλλους δύο.

Τα σενάρια που μελετήσαμε μπορούν να εφαρμοστούν αντίστοιχα και σε άλλες περιπτώσεις. Η Αστική, Προαστιακή και Αγροτική Περιοχή μπορούν να δώσουν πληροφορίες για την συμπεριφορά των μοντέλων στον πραγματικό κόσμο αλλά επιτρέπει και να μελετηθούν περιπτώσεις χρήσιμων και λιγότερο χρήσιμων κόμβων. Με την πληθώρα εφαρμογών της Ομοσπονδιακής Μάθησης ένα τέτοιο οικοσύστημα μπορεί να χρησιμοποιηθεί και με άλλα δεδομένα ή για άλλα μοντέλα. Για παράδειγμα σε σενάριο μίας Έξυπνης Πόλης, μπορεί αντί για Δημόσια Ασφάλεια (πυρκαγιές, πλημμύρες, σεισμοί) να εξετασθεί μια περίπτωση αυτόματης ρύθμισης της κυκλοφορίας στους δρόμους. Το μοντέλο που χρησιμοποιήθηκε (MobileNetV3) είναι ένα πολύ μικρό σε μέγεθος και υπολογιστική πολυπλοκότητα και άρα ιδανικό για εγκατάσταση σε οποιαδήποτε συσκευή. Επιπλέον, με την τεχνική εξαγωγής δεδομένων, το ακριβό υπολογιστικά κομμάτι ανάλυσης των εικόνων γίνεται μία μόνο φορά, επιτρέποντας φθηνές επαναλήψεις στην διαδικασία της Ομοσπονδιακής Μάθησης. Αντίστοιχα, με άλλα μοντέλα (όχι εικόνων) μπορούν να μελετηθούν άλλα φαινόμενα. Για παράδειγμα, η μελέτη του κινδύνου πυρκαγιών μέσω μετρήσεων θερμοκρασίας ή η πρόγνωση σεισμών μέσω δεδομένων από δίκτυα σεισμικής παρακολούθησης, αποτελούν δύο διαφορετικές προσεγγίσεις στο πρόβλημα το οποίο εμείς εξετάσαμε με εικόνες. Στο πλαίσιο των εικόνων - δεδομένων, προφανώς με ελάχιστες αλλαγές μπορούν να δοκιμασθούν και άλλα μοντέλα, όπως τα μοντέλα EfficientNet. Ανάλογα την εφαρμογή μπορεί να κάνουν καλύτερη εξαγωγή χαρακτηριστικών από τις εικόνες και αξίζει να γίνει μελέτη για την κατάλληλη επιλογή.