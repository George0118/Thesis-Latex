\renewcommand{\abstractname}{Περίληψη}

\begin{abstract}
Σε αυτή τη διατριβή, αντιμετωπίζουμε το πρόβλημα της αντιστοίχισης κόμβων - εξυπηρετητών σε ένα περιβάλλον Ομοσπονδιακής Μάθησης(Federated Learning) πολλαπλών μοντέλων, με στόχο την μεγιστοποίηση της επίδοσής του συγκεντρωτικού/παγκόσμιου μοντέλου. Κάθε ένας από τους εξυπηρετητές εκπαιδεύει ένα ξεχωριστό μοντέλο, ενώ οι κόμβοι διαθέτουν διαφορετική πληροφορία, συνεπώς για κάθε έναν από τους εξυπηρετητές έχουν διαφορετική σημασία. Σε αυτό το πλαίσιο μελετάμε και κατασκευάζουμε αλγορίθμους αντιστοίχισης σε διάφορα σενάρια ώστε να γίνουν εμφανή τα πλεονεκτήματα και μειονεκτήματα καθενός από αυτούς. Προκειμένου να προσδοθεί μια ρεαλιστική εφαρμογή ενός τέτοιου προβλήματος μελετάμε σενάρια εντοπισμού φυσικών κινδύνων (πυρκαγιών, πλημμύρων, σεισμών), μέσω φωτογραφιών, σε ένα περιβάλλον Έξυπνης Πόλης - Δημόσιας Ασφάλειας. Για τους αλγορίθμους αντιστοίχισης καταφεύγουμε στη Θεωρία Παιγνίων (Game Theory), Ενισχυτική Μάθηση (Reinforcement Learning) και στη Μετανοητική Μάθηση (Regret Learning), και στη συνέχεια εκτελούμε τη διαδικασία της Ομοσπονδιακής Μάθησης για να λάβουμε και να συγκρίνουμε τα αποτελέσματά μας.    
  
\begin{keywords}
  Ομοσπονδιακή Μάθηση, Ομοσπονδιακή Μάθηση Πολλαπλών Μοντέλων, Αλγόριθμοι Αντιστοίχισης, Θεωρία Παιγνίων, Ενισχυτική Μάθηση, Μετανοητική Μάθηση, Νευρωνικά Δίκτυα, Συνασπισμοί, Διαμόρφωση Συμμετοχής Κόμβων.
\end{keywords}

\end{abstract}


\begin{abstracteng}
  In this thesis, we address the problem of node-server matching in a multi-model Federated Learning environment, aiming to maximize its performance. Each server trains a separate model, while the nodes possess different information, and therefore, hold varying significance for each server. Within this environment, we study and design matching algorithms under various scenarios to highlight the advantages and disadvantages of each. To provide a realistic application of such a problem, we examine scenarios related to the detection of natural hazards (fires, floods, earthquakes) through photographs in a Smart City - Public Safety environment. For the matching algorithms, we utilize Game Theory, Reinforcement Learning, and Regret Learning and subsequently perform the Federated Learning process to obtain and compare our results.

\begin{keywordseng}
    Federated Learning, Multi-Model Federated Learning, Matching Algorithms, Game Theory, Reinforcement Learning, Regret Learning, Neural Networks, Coalitions, Configuration of Nodes' Participation
\end{keywordseng}
\end{abstracteng}

\begin{acknowledgements}

TBD


\bigskip

\bigskip

\bigskip

\end{acknowledgements}