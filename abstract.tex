\renewcommand{\abstractname}{Περίληψη}
\iffalse
\begin{abstract}
Σε αυτή τη διατριβή, αντιμετωπίζουμε το πρόβλημα της αποτελεσματικής κατανομής πόρων σε ετερογενή και πολυεπίπεδα ασύρματα δίκτυα επικοινωνιών και κινητών υπολογισμών με έναν ολιστικό, αλλά κατανεμημένο, τρόπο. Προς την κατεύθυνση των αυτο-οργανωμένων δικτύων, όπου κάθε δικτυακή οντότητα λαμβάνει αυτόνομες αποφάσεις σχετικά με τη χρήση και την κατανομή των πόρων της, αναπτύσσουμε νέα πλαίσια τα οποία αντικατοπτρίζουν τις αλληλεξαρτήσεις μεταξύ των συμπεριφορών, των αλληλεπιδράσεων και των αποφάσεων των διαφορετικών διακτυακών οντοτήτων, λαμβάνοντας υπόψιν τους διαφορετικούς ή/και αντικρουόμενους στόχους τους, καθώς και την ύπαρξη ελλιπούς πληροφόρησης μεταξύ τους. Προκειμένου να προσδοθεί μια τέτοια ρεαλιστική χρειά στη μοντελοποίηση των προβλημάτων κατανομής πόρων, καταφεύγουμε στη Θεωρία Παιγνίων (Game Theory) και στη Θεωρία Συμβολαίων (Contract Theory), οι οποίες με τη σειρά τους οδηγούν σε μεθοδολογίες και αλγόριθμους χαμηλής πολυπλοκότητας για την επίλυση των αντίστοιχων προβλημάτων.    
  
\begin{keywords}
  Πολλαπλή Πρόσβαση Διαίρεσης Ρυθμού, Βαθιά Ενισχυτική Μάθηση, Συστήματα Πολλαπλών Πρακτόρων, Δίκτυα 6\textsuperscript{ης} γενιάς, Ενεργειακή Απόδοση, Κατανομή πόρων 
\end{keywords}

\end{abstract}


\begin{abstracteng}
In this thesis, we tackle the problem of efficient resource allocation in heterogeneous and multi-tier wireless communication and mobile computing networks in a holistic, though distribu-ted, manner. In the direction of self-organizing networks, where each network entity makes autonomous decisions regarding its resource utilization and allocation, we develop novel frame-works that capture the interdependencies between different network entities' behaviors, interac-tions, and decisions by accounting for their different and/or conflicting objectives and the existen-ce of incompleteness of information between them. To provide such a real-life spirit in modeling the resource management problems, we resort to Game Theory and Contract Theory, which result in low-complexity methodologies and algorithms for solving the formulated problems. 

\begin{keywordseng}
    Resource Allocation, ...
\end{keywordseng}
\end{abstracteng}
\fi

\begin{acknowledgements}

TBD


\bigskip

\bigskip

\bigskip

\end{acknowledgements}