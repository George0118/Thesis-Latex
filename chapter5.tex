\chapter{Συμπεράσματα}

Έπειτα από μελέτη των διαφόρων σεναρίων, αλγορίθμων και αποτελεσμάτων μπορούμε να καταλήξουμε σε κάποια γενικά συμπεράσματα από την διαδικασία αυτής της διπλωματικής εργασίας.

Η Ομοσπονδιακή Μάθηση είναι ένας μηχανισμός που μας επιτρέπει να εξάγουμε πληροφορία από δεδομένα, χωρίς να αποκτούμε πρόσβαση σε αυτά. Έτσι διασφαλίζουμε την ιδιωτικότητα, που είναι απαραίτητη ιδιαίτερα σε συγκεκριμένους τομείς, πετυχαίνοντας παράλληλα την εκπαίδευση των μοντέλων που χρειαζόμαστε. Στην περίπτωσή μας, είχαμε τρία μοντέλα, τα οποία εκπαιδεύονταν από τους κόμβους του περιβάλλοντός μας. Ο κάθε ένας κόμβος διέθετε μια συγκεκριμένη ποσότητα και τύπο δεδομένων και άρα θα είχε διαφορετική σημασία για κάθε μοντέλο - εξυπηρετητή. Αυτό αποδίδεται στους αλγορίθμους μας με τις συναρτήσεις χρησιμότητας. Τις συναρτήσεις αυτές φαίνεται να διαχειρίζεται καλύτερα ο αλγόριθμος Θεωρίας Παιγνίων, ο οποίος καταφέρνει να κάνει τις καλύτερες αντιστοιχίσεις, κρατώντας τις χρησιμότητες υψηλότερα από τους υπόλοιπους αλγορίθμους. Συνεπώς, έπειτα είναι ευθύνη μας να χτίσουμε μία αντιπροσωπευτική συνάρτηση χρησιμότητας που να αντικατοπτρίζει ορθά το πρόβλημα και τις παραμέτρους του. 

Όπως είδαμε ακόμη και με μία σχετικά απλή συνάρτηση χρησιμότητας, ο αλγόριθμος Θεωρίας Παιγνίων προσπαθώντας να την μεγιστοποιήσει καταφέρνει να πετύχει υψηλή ροή δεδομένων, χαμηλή ενέργεια μετάδοσης και καλύτερη επίδοση στην Ομοσπονδιακή Μάθηση έναντι των υπόλοιπων αλγορίθμων αντιστοίχισης. Παρολαυτά, και οι άλλοι αλγόριθμοι πετυχαίνουν πολύ καλά αποτελέσματα με τον αλγόριθμο Ενισχυτικής Μάθησης ως προς χρησιμότητα κόμβων να ακολουθεί σε επίδοση. Είναι επίσης σημαντικό να επαναλάβουμε πως και να γίνει κάποιο λάθος στην αντιστοίχιση και να υπάρχει κάποιος μη χρήσιμος κόμβος στον συνασπισμό ενός εξυπηρετητή, ο μηχανισμός της ανάθεσης βαρών για την διαδικασία της Ομοσπονδιακής Μάθησης επιτρέπει να μην επηρεάσει ιδιαίτερα την επίδοση. Μία επιπλέον βελτίωση θα ήταν ένας εξυπηρετητής να μην λαμβάνει υπόψη του κάποιον κόμβο ο οποίος έχει πολύ μικρό βάρος έτσι ώστε να μην καταναλωθούν πόροι για την τοπική εκπαίδευση και μετάδοση των παραμέτρων του μοντέλου. Από την άλλη πλευρά με μη βέλτιστη αντιστοίχιση δεν αξιοποιείται πληροφορία που για κάποιον άλλο εξυπηρετητή θα ήταν χρήσιμη. Έτσι, ξεχωρίζουμε την ικανότητα του αλγρίθμου Θεωρίας Παιγνίων να κάνει βέλτιστη αντιστοίχιση σχεδόν σε κάθε ένα από τα σενάρια που του ανατέθηκαν.