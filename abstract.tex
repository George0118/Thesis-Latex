\renewcommand{\abstractname}{Περίληψη}

\begin{abstract}

\vspace{-2cm}

Με την ευρύτερη εφαρμογή της Ομοσπονδιακής Μάθησης (Federated Learning) η επιλογή κόμβων από εξυπηρετητές αποτελεί κρίσιμο πρόβλημα και ειδικότερα σε περιβάλλοντα όπου συνυπάρχουν πολλαπλοί εξυπηρετητές. Μέχρι τώρα μελέτες επικεντρώνονται στην επιλογή κόμβων από έναν εξυπηρετητή, για την εκπαίδευση του παγκόσμιου μοντέλου του. Σε αυτή την εργασία προεκτείνουμε την λογική αυτή και αντιμετωπίζουμε την αντιστοίχιση κόμβων - εξυπηρετητών σε ένα περιβάλλον Ομοσπονδιακής Μάθησης πολλαπλών εξυπηρετητών, με στόχο την μεγιστοποίηση των χρησιμοτήτων κόμβων και εξυπηρετητών και ως προέκταση της επίδοσης των συγκεντρωτικών μοντέλων. Κάθε ένας από τους εξυπηρετητές εκπαιδεύει ένα ξεχωριστό μοντέλο, ενώ οι κόμβοι διαθέτουν διαφορετική πληροφορία, συνεπώς για κάθε εξυπηρετητή έχουν διαφορετική σημασία. Σε αυτό το πλαίσιο κατασκευάζουμε και μελετάμε αλγορίθμους αντιστοίχισης σε διάφορα σενάρια ώστε να γίνουν εμφανή τα πλεονεκτήματα και μειονεκτήματα τους. Ως εφαρμογή ενός τέτοιου προβλήματος μελετάμε σενάρια εντοπισμού φυσικών κινδύνων (πυρκαγιών, πλημμυρών, σεισμών), μέσω φωτογραφιών, σε ένα περιβάλλον Έξυπνης Πόλης - Δημόσιας Ασφάλειας.

Για τους αλγορίθμους αντιστοίχισης καταφεύγουμε στη Θεωρία Παιγνίων (Game Theory), Ενισχυτική Μάθηση (Reinforcement Learning) και στη Μετανοητική Μάθηση (Regret Learning), και στη συνέχεια εκτελούμε τη διαδικασία της Ομοσπονδιακής Μάθησης για να λάβουμε και να συγκρίνουμε τα αποτελέσματά μας. Για τους αλγορίθμους Μετανοητικής Μάθησης που προτείνουμε, δίνουμε επιπλέον ελευθερία στους κόμβους μας να διαμορφώσουν την συμμετοχή τους στην διαδικασία της Ομοσπονδιακής Μάθησης ανάλογα με το συμφέρον τους. Μέσω πειραμάτων και συγκρίσεων των διαφόρων αλγορίθμων, καταλήγουμε πως ο αλγόριθμος Θεωρίας Παιγνίων υπερέχει των αλγορίθμων Ενισχυτικής Μάθησης και της Τυχαίας Αντιστοίχισης, επιτυγχάνοντας υψηλότερες χρησιμότητες για τους κόμβους και εξυπηρετητές και καλύτερες επιδόσεις για τα παγκόσμια μοντέλα. Αντίστοιχα, οι αλγόριθμοι Μετανοητικής Μάθησης, με την επιπλέον ιδιότητά τους να διαμορφώνουν την συμμετοχή των κόμβων στην Ομοσπονδιακή Μάθηση, παρουσιάζουν ακόμη καλύτερες χρησιμότητες, πλησιάζοντας την απόδοση των παγκόσμιων μοντέλων του αλγορίθμου Θεωρίας Παιγνίων με χρήση πολύ λιγότερων δεδομένων και ενέργειας.

\vspace{-0.5cm}
  
\begin{keywords}
  Ομοσπονδιακή Μάθηση, Ομοσπονδιακή Μάθηση Πολλαπλών Μοντέλων, Αλγόριθμοι Αντιστοίχισης, Θεωρία Παιγνίων, Ενισχυτική Μάθηση, Μετανοητική Μάθηση, Νευρωνικά Δίκτυα, Συνασπισμοί, Διαμόρφωση Συμμετοχής Κόμβων.
\end{keywords}

\end{abstract}

\begin{abstracteng}

\vspace{-2cm}

With the development and broader application of Federated Learning, the problem of node selection by the corresponding server becomes more prominent. Specifically, in environments where more than one server coexists, such a process is critical. So far, studies have focused on node selection from the perspective of a single server to train its global model. In this thesis, we extend this logic and address the problem of node - server assignment in a Federated Learning environment with multiple servers and models. The goal is to maximize the utilities of both nodes and servers and, consequently, the performance of the global models. Each server trains a distinct model, while the nodes hold different data, and therefore have varying significance/importance for each server. Within this framework, we study and design matching algorithms across various scenarios to highlight the advantages and disadvantages of each. To provide a realistic application of such a problem, we examine scenarios involving the detection of natural hazards (fires, floods, earthquakes) through images in a Smart City – Public Safety environment.

For the matching algorithms, we construct algorithms based on Game Theory, Reinforcement Learning, and Regret Learning. We then execute the Federated Learning process to obtain and compare our results. For the proposed Regret Learning algorithms, we offer additional flexibility to the nodes, allowing them to adjust their participation and the resources they allocate to the Federated Learning process based on their interests (utility). Through experiments and comparisons of the various algorithms, we conclude that the Game Theory matching algorithm outperforms the Reinforcement Learning algorithms and Random Matching, achieving higher utilities for both nodes and servers as well as better performance for the global models. Similarly, the Regret Learning algorithms, with their additional ability to shape node participation in Federated Learning, demonstrate even better utilities, approaching the performance of the global models produced by the Game Theory algorithm while using significantly less data and energy.

\begin{keywordseng}
    Federated Learning, Multi-Model Federated Learning, Matching Algorithms, Game Theory, Reinforcement Learning, Regret Learning, Neural Networks, Coalitions, Configuration of Nodes' Participation
\end{keywordseng}
\end{abstracteng}

\begin{acknowledgements}

Θα ήθελα σε αυτό το κομμάτι να εκφράσω τις ευχαριστίες μου σε όλους όσους μου παραστάθηκαν στη διάρκεια των σπουδών μου στο Εθνικό Μετσόβειο Πολυτεχνείο και ιδιαίτερα τον τελευταίο χρόνο στο πλαίσιο εκτέλεσης της διπλωματικής μου εργασίας.

Ευχαριστώ τον Δρ. Συμεών Παπαβασιλείου, Καθηγητή στο Εθνικό Μετσόβειο Πολυτεχνείο στον Τομέα Επικοινωνιών, Ηλεκτρονικής και Συστημάτων Πληροφορικής αρχικά για την διδασκαλία του που μου κέντρισε το ενδιαφέρον και στη συνέχεια για την καθοδήγηση και συνεργασία στη διπλωματική μου και τις συμβουλές του για τις μεταπτυχιακές μου σπουδές.

Ευχαριστώ, επίσης την Δρ. Ειρήνη Ελένη Τσιροπούλου, Καθηγήτρια στο Πολιτειακό Πανεπιστήμιο της Αριζόνα για την συνεργασία που είχαμε κατά τη διάρκεια της διπλωματικής μου. Η καθοδήγηση, οργάνωση και υποστήριξη από πλευράς της ήταν ανεκτίμητη και είμαι ευγνώμων που είχα την ευκαιρία να συνεργαστώ μαζί της και με άλλους διδάκτορες στο εργαστήριό της.

Ευχαριστώ την Δρ. Μαρία Διαμαντή, Μεταδιδακτορική Ερευνήτρια στο Εθνικό Μετσόβειο Πολυτεχνείο στον Τομέα Επικοινωνιών, Ηλεκτρονικής και Συστημάτων Πληροφορικής για την βοήθειά της στην επίλυση αποριών και προβλημάτων αλλά και στην ομαλή και γρήγορη εξοικείωσή μου με το αντικείμενο και τους στόχους της διπλωματικής μου εργασίας.

Ευχαριστώ τον Δρ. Γεώργιο Γκούμα, Καθηγητή στο Εθνικό Μετσόβειο Πολυτεχνείο στον Τομέα Τεχνολογίας Πληροφορικής και Υπολογιστών για τις συμβουλές του κατά τη διάρκεια των σπουδών μου αλλά και γιατί με τη διδασκαλία του με βοήθησε να έρθω σε επαφή με τον τομέα των Λειτουργικών Συστημάτων και Παράλληλων Συστημάτων.

Για τις ανάγκες της διπλωματικής μου εργασίας χρειάστηκαν σημαντικοί υπολογιστικοί πόροι, τόσο από πλευράς επεξεργαστών και μνήμης, όσο και καθαρού χρόνου. Για την υποστήριξη σε αυτόν τον τομέα θα ήθελα να ευχαριστήσω τους Δρ. Δημήτριο Ντελλή και Κυριάκο Γκίνη για την διαθεσιμότητά τους και την άμεση απόκρισή τους για οποιαδήποτε τεχνικά ζητήματα προέκυψαν στο υπολογιστικό σύστημα HPC ARIS του GRNET (\href{https://www.hpc.grnet.gr/}{\underline{HPC ARIS}}), στο οποίο έγινε το μεγαλύτερο μέρος πειραμάτων της διπλωματικής μου εργασίας. Επιπλέον, θα ήθελα να ευχαριστήσω τον Δρ. Μίνωα Αξενίδη και Δρ. Δάκη Παυλίδη για τη δυνατότητα που μου έδωσαν να χρησιμοποιήσω το υπολογιστικό σύστημα (cluster) Gauss στο Ινστιτούτο Πυρηνικής \& Σωματιδιακής Φυσικής του Εθνικού Κέντρου Έρευνας Φυσικών Επιστημών Δημόκριτος (\href{https://www.demokritos.gr/el/institute/%CE%B9%CE%BD%CF%83%CF%84%CE%B9%CF%84%CE%BF%CF%8D%CF%84%CE%BF-%CF%80%CF%85%CF%81%CE%B7%CE%BD%CE%B9%CE%BA%CE%AE%CF%82-%CF%83%CF%89%CE%BC%CE%B1%CF%84%CE%B9%CE%B4%CE%B9%CE%B1%CE%BA%CE%AE%CF%82-%CF%86/}{\underline{INP NCSR Demokritos}}) για την εκτέλεση επιπλέον πειραμάτων της διπλωματικής μου εργασίας.

Ευχαριστώ την οικογένειά μου που με βοήθησε καθ' όλη τη διάρκεια των σπουδών μου, διότι χωρίς αυτούς δεν θα είχα καταφερει να φτάσω έως εδώ.

Ευχαριστώ τους φίλους μου, Αναστασία-Χριστίνα, Νικόλα, Αθηνά, Ναταλία, Γιώργο, Φίλιππο, για την υποστήριξή τους και για τις όμορφες στιγμές και αναμνήσεις που θα κρατήσω για πάντα.


\bigskip

\bigskip

\bigskip

\end{acknowledgements}