\chapter{Εισαγωγή}
\vspace{-0.5cm}
\paragraph{}Τα τελευταία χρόνια όπου αναπτύσεται με ραγδαίους ρυθμούς ο κλάδος της Τεχνητής Νοημοσύνης και των Νευρωνικών Δικτύων έχει μεγαλώσει αντίστοιχα η ανάγκη για μεγάλους όγκους δεδομένων, για την εκπαίδευση των εκάστοτε μοντέλων. Ο καθένας μας, μέσω των ηλεκτρονικών συσκευών, που πλέον είναι εκτενώς διαθέσιμες (κινητά τηλέφωνα, υπολογιστές, IoT συσκευές), αλλά και με τη βοήθεια του διαδικτύου, καταναλώνει καθημερινά έναν τεράστιο όγκο δεδομένων, όπως εικόνες, βίντεο, μουσική, ηχητικά μηνύματα και απλό κείμενο. Το σημαντικό για όλα αυτά τα δεδομένα είναι πως είναι φτιαγμένα από ανθρώπους για ανθρώπους και συνεπώς αποτελούν πολύ καλή πληροφορία για την εκπαίδευση μοντέλων για πληθόρα εφαρμογών. Επιπλέον, η ιδιομορφία του κάθε ανθρώπου και η διαφορά στον χαρακτήρα του, αντικατοπτρίζεται στην καθημερινότητά του και άρα και στην αλληλεπίδρασή του στο διαδίκτυο ή στις συσκευές του. Συνεπώς, τα δεδομένα αυτά αποκτούν μια επιπλέον αξία, επιτρέποντας σε μοντέλα εκπαιδευμένα σε αυτά να αναλύσουν τις ιδιομορφίες και διαφορές στη συμπεριφορά των ανθρώπων, δημιουργώντας ένα ακόμα πιο αληθοφανές και ανθρωποειδές αποτέλεσμα. 
\vspace{-1cm}
\paragraph{}Από την άλλη πλευρά, τα θέματα ιδιωτικότητας και προσωπικών δεδομένων αποτελούν επίσης ένα σύγχρονο πρόβλημα. Είναι προφανές πως δεν θα θέλαμε κάποιος που μας προσφέρει μια υπηρεσία στο κινητό μας να έχει πρόσβαση στα προσωπικά δεδομένα μας (φωτογραφίες, μηνύματα κ.α.). Το πρόβλημα της ιδιωτικότητας στο διαδίκτυο δεν θεωρούταν τόσο σημαντικό πριν ακόμα και από μία δεκαετία, αλλά με την πρόσφατη εκθετική αύξηση χρησιμοποιήσης του διαδικτύου και των υπηρεσιών που αυτό προσφέρει, έχει έρθει στο προσκήνιο ως μια απαραίτητη προϋπόθεση από πλευράς του κόμβου. Αντίστοιχα, έπειτα από σκάνδαλα για πώληση προσωπικών δεδομένων κόμβων από μικρές και μεγάλες εταιρείες ανεξαιρέτως, έχουν τεθεί αυστηροί κανόνες και νόμοι (είτε ανά χώρα, είτε στην Ευρώπη από την Ε.Ε.) για την προστασία των κόμβων.
\vspace{-1cm}
\paragraph{}Η προστασία των προσωπικών δεδομένων, λοιπόν τήθεται ως ήθικο εμπόδιο στην αξιοποίηση του τεράστιου αυτού όγκου πληροφορίας. Έτσι ήταν πολύ σημαντικό να βρεθεί ένας τρόπος ο οποίος θα επιτρέπει να επωφεληθούμε από τα δεδομένα του κάθε κόμβου, χωρίς όμως να παραβιάζεται η ιδιωτικότητά του. Την λύση αυτή έδωσε η Ομοσπονδιακή Μάθηση, η οποία όπως θα μελετήσουμε και στη συνέχεια λύνει τα δύο αυτά προβλήματα με κομψό τρόπο.

\section{Βιβλιογραφική Επισκόπηση}

\paragraph{}Κατά τη διάρκεια της παρούσας διπλωματικής εργασίας, θα αναφερθούμε σε μια σειρά από βασικούς όρους και έννοιες που αποτελούν βάση των αναλύσεων και των εφαρμογών που θα παρουσιαστούν. Οι έννοιες αυτές είναι κρίσιμες για την κατανόηση των θεμάτων που θα εξεταστούν και γι' αυτόν τον λόγο είναι απαραίτητο να αναφερθούν από την αρχή. Οι βασικοί αυτοί όροι παρουσιάζονται παρακάτω:

\begin{enumerate}
    \item Συνάρτηση Χρησιμότητας: Στα οικονομικά και στη θεωρία παιγνίων, η συνάρτηση χρησιμότητας είναι μια μαθηματική αναπαράσταση των προτιμήσεων ενός παίκτη. Αποδίδει μια αριθμητική τιμή σε κάθε πιθανό αποτέλεσμα, υποδεικνύοντας το σχετικό επίπεδο ικανοποίησης ή οφέλους που ο παίκτης αποκομίζει από αυτό το αποτέλεσμα. Όσο υψηλότερη είναι η τιμή, τόσο μεγαλύτερη είναι η ικανοποίηση.

    \item Θεωρία Παιγνίων: Η θεωρία παιγνίων είναι η μελέτη των στρατηγικών αλληλεπιδράσεων μεταξύ ατόμων. Παρέχει εργαλεία για την ανάλυση καταστάσεων όπου οι επιλογές των ατόμων επηρεάζουν τα αποτελέσματα των άλλων. Με βάση την αλληλεπίδραση στο κοινό περιβάλλον, καθώς και τις προσωπικές προτιμήσεις κάθε παίκτη, στόχος είναι να πάρει ο καθένας την καλύτερη δυνατή απόφαση.

    \item Παιχνίδι Αντιστοίχισης: Το παιχνίδι αντιστοίχισης είναι ένας κλάδος της θεωρίας παιγνίων όπου οι παίκτες έχουν ως στόχο να αντιστοιχηθούν σε κάποιον άλλο παίκτη ή ομάδα, με σκοπό να βελτιστωποιήσουν το κέρδος τους. Τέτοια παιχνίδια συχνά χρησιμοποιούνται για την ανάλυση προβλημάτων κατανομής πόρων.

    \item Παιχνίδι Συμμαχίας: Ένα παιχνίδι συμμαχίας, ή συνεργατικό παιχνίδι, είναι ένας κλάδος της θεωρίας παιγνίων όπου οι παίκτες μπορούν να σχηματίσουν συμμαχίες (συνασπισμούς) για να επιτύχουν καλύτερα αποτελέσματα συλλογικά. Η εστίαση είναι στο πώς να επιτευχθεί καλύτερο συλλογικό, αλλά και μεμωνομένο για κάθε παίκτη, αποτέλεσμα.

    \item Ενισχυτική Μάθηση: Η Ενισχυτική Μάθηση (RL) είναι ένας τύπος μηχανικής μάθησης όπου ένας πράκτορας μαθαίνει να παίρνει αποφάσεις εκτελώντας ενέργειες σε ένα περιβάλλον για να μεγιστοποιήσει τη αμοιβή του. Ο πράκτορας δοκιμάζει τις δυνατές ενέργειές του σε κάθε επανάληψη, προσπαθώντας να αποφανθεί ποιές από αυτές του προσφέρουν τις καλύτερες ανταμοιβές. Έτσι προσπαθεί να πάρει απόφαση για την βέλτιστη ή τις βέλτιστες ενέργειες που μπορεί να εκτελέσει.

    \item Εκτός Πολιτικής Αλγόριθμος: Στην ενισχυτική μάθηση, ο όρος "εκτός πολιτικής" αναφέρεται σε ένα τύπο μάθησης όπου ο πράκτορας μπορεί να μάθει για μια βέλτιστη ή στόχο πολιτική ενώ ακολουθεί μια διαφορετική συμπεριφορική πολιτική. Αυτό σημαίνει ότι ο πράκτορας μπορεί να μάθει από ενέργειες που δεν λαμβάνει απαραίτητα υπό την τρέχουσα πολιτική. Οι αλγόριθμοι εκτός πολιτικής χρησιμοποιούν εμπειρίες που δημιουργούνται από οποιαδήποτε πολιτική, όχι μόνο από αυτή που βελτιστοποιείται. Τα κύρια χαρακτηριστικά περιλαμβάνουν τη διάκριση μεταξύ της συμπεριφορικής πολιτικής, την οποία ακολουθεί ο πράκτορας για να δημιουργεί ενέργειες και να συλλέγει εμπειρίες, και της πολιτικής στόχου, την οποία ο πράκτορας επιδιώκει να βελτιστοποιήσει. Αυτή η προσέγγιση επιτρέπει στον πράκτορα να εξερευνά το περιβάλλον χρησιμοποιώντας μια συμπεριφορική πολιτική που ενθαρρύνει την εξερεύνηση, ενώ μαθαίνει μια βέλτιστη πολιτική που εκμεταλλεύεται τις γνωστές ανταμοιβές.

    \item Μάθηση με Μείωση της Μεταμέλειας (Μετανοητική Μάθηση): Η μάθηση με μείωση της μεταμέλειας είναι μια στρατηγική στη θεωρία παιγνίων και την μηχανική μάθηση όπου οι παίκτες προσαρμόζουν τις ενέργειές τους βάσει της προηγούμενης απόδοσης για να ελαχιστοποιήσουν τη μεταμέλεια. Η μεταμέλεια μετρά τη διαφορά μεταξύ της πραγματικής αμοιβής που έλαβε ο παίκτης και της αμοιβής της κάθε δυνατής ενέργειας. Ο στόχος είναι να μάθει ο παίκτης τις καλύτερες για αυτόν ενέργειες με βάση το πόσο μετανιώνει όταν τις εκτέλεσε ή δεν τις εκτέλεσε.

    \item Ενεργή Μάθηση: Η ενεργή μάθηση στη μηχανική μάθηση είναι ένα μοντέλο μάθησης όπου ο αλγόριθμος ενημερώνει τη γνώση του διαδοχικά καθώς φτάνουν νέα δεδομένα, επιτρέποντάς του να προσαρμόζεται σε πραγματικό χρόνο σε αλλαγές. Σε αντίθεση με την παραδοσιακή μάθηση κατά παρτίδες, η οποία επεξεργάζεται ολόκληρο το σύνολο δεδομένων ταυτόχρονα, η ενεργή μάθηση μαθαίνει συνεχώς από μεμονωμένα δεδομένα ή μικρές παρτίδες, καθιστώντας την πιο αποδοτική από άποψη μνήμης και επεκτάσιμη. Αυτή η προσέγγιση είναι ιδιαίτερα χρήσιμη σε δυναμικά περιβάλλοντα, όπως τα συστήματα συστάσεων, οι χρηματοπιστωτικές αγορές και η ανάλυση σε πραγματικό χρόνο, όπου τα δεδομένα εξελίσσονται γρήγορα. Ένα κοινό παράδειγμα είναι η Στοχαστική Κατάβαση Κλίσης (SGD), η οποία ενημερώνει τις παραμέτρους του μοντέλου σταδιακά με κάθε νέο δεδομένο.

    \item Ταξινόμηση Εικόνας: Η ταξινόμηση εικόνας είναι ένα πρόβλημα όρασης υπολογιστών όπου ένας αλγόριθμος αποδίδει μια ετικέτα ή κατηγορία σε μια εικόνα βάσει του οπτικού της περιεχομένου. Περιλαμβάνει την εκπαίδευση μοντέλων, συχνά νευρωνικών δικτύων, για την αναγνώριση και κατηγοριοποίηση αντικειμένων, σκηνών ή άλλων μοτίβων στις εικόνες.

    \item Νευρωνικά Δίκτυα / Αποδοτικά Νευρωνικά Δίκτυα: Τα νευρωνικά δίκτυα είναι μια κατηγορία μοντέλων μηχανικής μάθησης εμπνευσμένη από τη δομή και τη λειτουργία του ανθρώπινου εγκεφάλου. Αποτελούνται από διασυνδεδεμένα στρώματα κόμβων (νευρώνες) που επεξεργάζονται δεδομένα με ιεραρχικό τρόπο για την εκτέλεση καθηκόντων όπως ταξινόμηση, παλινδρόμηση και αναγνώριση μοτίβων. Τα αποδοτικά νευρωνικά δίκτυα είναι βελτιστοποιημένες εκδόσεις σχεδιασμένες να επιτυγχάνουν υψηλή απόδοση με μειωμένους υπολογιστικούς πόρους, καθιστώντας τα κατάλληλα για εκπαίδευση και αξιοποίηση σε συσκευές με περιορισμένη ισχύ επεξεργασίας, όπως κινητά τηλέφωνα.

    \item Μοντέλο και Στρώματα Μοντέλου: Ένα μοντέλο στη μηχανική μάθηση είναι μια μαθηματική αναπαράσταση ενός συστήματος που χρησιμοποιείται για την πρόβλεψη ή τη λήψη αποφάσεων βάσει εισαγόμενων δεδομένων. Τα μοντέλα μπορεί να κυμαίνονται από απλές γραμμικές παλινδρομήσεις έως πολύπλοκα νευρωνικά δίκτυα. Τα στρώματα μοντέλου αναφέρονται στα ατομικά δομικά στοιχεία ενός νευρωνικού δικτύου. Κάθε στρώμα αποτελείται από νευρώνες που εφαρμόζουν συγκεκριμένους μετασχηματισμούς στα εισαγόμενα δεδομένα. Κοινά είδη στρώσεων περιλαμβάνουν τα συνελικτικά στρώματα (για την ανίχνευση χωρικών χαρακτηριστικών), τα στρώματα υποδειγμάτων (για τη μείωση του μεγέθους) και τα πλήρως συνδεδεμένα στρώματα (για την ενσωμάτωση χαρακτηριστικών).    

    \item Προ-εκπαιδευμένο Μοντέλο Μηχανικής Μάθησης: Ένα προ-εκπαιδευμένο μοντέλο στη μηχανική μάθηση είναι ένα μοντέλο που έχει ήδη εκπαιδευτεί σε ένα μεγάλο σύνολο δεδομένων και στη συνέχεια προσαρμόζεται ή βελτιστοποιείται για να εκτελεί συγκεκριμένα καθήκοντα. Αυτή η προσέγγιση αξιοποιεί τη γνώση που έχει αποκτήσει το μοντέλο κατά την αρχική φάση της εκπαίδευσης για να βελτιώσει την απόδοση και να μειώσει το χρόνο και τους πόρους που απαιτούνται για την εκπαίδευση σε νέες εργασίες.

    Αρχική Εκπαίδευση: Το μοντέλο εκπαιδεύεται πρώτα σε ένα μεγάλο, γενικό σύνολο δεδομένων. Για παράδειγμα, στην επεξεργασία φυσικής γλώσσας (NLP), μοντέλα όπως το BERT ή το GPT εκπαιδεύονται σε τεράστιες ποσότητες κειμένων από το διαδίκτυο.
    
    Μεταφορά Γνώσης: Το προ-εκπαιδευμένο μοντέλο έχει ήδη μάθει χρήσιμα χαρακτηριστικά και μοτίβα από την αρχική του εκπαίδευση. Αυτά τα χαρακτηριστικά μπορούν να μεταφερθούν σε νέες εργασίες, συχνά με αποτέλεσμα καλύτερη απόδοση.
    
    Αποδοτικότητα: Η χρήση ενός προ-εκπαιδευμένου μοντέλου μπορεί να μειώσει σημαντικά τους υπολογιστικούς πόρους και το χρόνο που απαιτείται για την εκπαίδευση, καθώς το μοντέλο δεν χρειάζεται να μάθει από την αρχή.
    
    Απόδοση: Τα προ-εκπαιδευμένα μοντέλα συχνά επιτυγχάνουν υψηλότερη απόδοση σε συγκεκριμένες εργασίες, καθώς αξιοποιούν τις πλούσιες αναπαραστάσεις χαρακτηριστικών που έχουν μάθει κατά την αρχική τους εκπαίδευση.

    \item Ακρίβεια: Η ακρίβεια είναι ένα μέτρο απόδοσης σε προβλήματα ταξινόμησης, που ορίζεται ως το ποσοστό των σωστών προβλέψεων σε σχέση με το συνολικό αριθμό των παραδειγμάτων. Η ακρίβεια χρησιμοποιείται συχνά για την αξιολόγηση της απόδοσης των αλγορίθμων μηχανικής μάθησης.

    \item Απώλεια: Η απώλεια είναι μια συνάρτηση που μετράει το πόσο καλά ή κακά αποδίδει ένα μοντέλο μηχανικής μάθησης. Αντιπροσωπεύει τη διαφορά μεταξύ των προβλέψεων του μοντέλου και των πραγματικών τιμών. Στόχος είναι η ελαχιστοποίηση της απώλειας κατά την εκπαίδευση του μοντέλου για να βελτιωθεί η ακρίβεια των προβλέψεων.

    \item Εξαγωγή Χαρακτηριστικών: Η εξαγωγή χαρακτηριστικών είναι μια κρίσιμη διαδικασία στη μηχανική μάθηση και την ανάλυση δεδομένων, όπου τα ακατέργαστα δεδομένα μετατρέπονται σε ένα σύνολο σχετικών γνωρισμάτων ή χαρακτηριστικών που μπορούν να χρησιμοποιηθούν για την εκπαίδευση ενός μοντέλου. Αυτή η διαδικασία αποσκοπεί στη μείωση της πολυπλοκότητας των δεδομένων διατηρώντας ταυτόχρονα τα βασικά τους μοτίβα και δομές. Η αποτελεσματική εξαγωγή χαρακτηριστικών βοηθά στη βελτίωση της απόδοσης των αλγορίθμων μηχανικής μάθησης εστιάζοντας στις πιο ενημερωτικές πτυχές των δεδομένων, επιτρέποντας έτσι στο μοντέλο να κάνει πιο ακριβείς προβλέψεις. Οι τεχνικές για την εξαγωγή χαρακτηριστικών μπορούν να διαφέρουν ανάλογα με τον τύπο των δεδομένων και το συγκεκριμένο πρόβλημα που αντιμετωπίζεται. Για παράδειγμα, στην επεξεργασία εικόνας, η εξαγωγή χαρακτηριστικών μπορεί να περιλαμβάνει την αναγνώριση ακμών, υφών ή σχημάτων μέσα σε μια εικόνα. Στην επεξεργασία φυσικής γλώσσας, μπορεί να περιλαμβάνει την εξαγωγή λέξεων-κλειδιών, n-γραμμάτων ή συντακτικών δομών από το κείμενο. Ο στόχος είναι να δημιουργηθεί μια απλοποιημένη αναπαράσταση των δεδομένων που διατηρεί τα σημαντικά τους χαρακτηριστικά, κάνοντάς τα πιο εύκολα για τα μοντέλα μηχανικής μάθησης να μάθουν και να γενικεύσουν καλά σε νέα, άγνωστα δεδομένα.

    \item Υπερπροσαρμογή: Η υπερπροσαρμογή είναι ένα συνηθισμένο πρόβλημα στη μηχανική μάθηση, όπου ένα μοντέλο μαθαίνει τα δεδομένα εκπαίδευσης πολύ καλά, καταγράφοντας θόρυβο και ανωμαλίες μαζί με τα υποκείμενα μοτίβα. Αυτό έχει ως αποτέλεσμα ένα μοντέλο που αποδίδει εξαιρετικά καλά στα δεδομένα εκπαίδευσης, αλλά άσχημα σε νέα, άγνωστα δεδομένα. Η υπερπροσαρμογή εμφανίζεται όταν το μοντέλο είναι πολύπλοκο σε σχέση με την ποσότητα των δεδομένων εκπαίδευσης, συχνά χαρακτηριζόμενο από την ύπαρξη υπερβολικά πολλών παραμέτρων. Αυτή η υπερβολική πολυπλοκότητα επιτρέπει στο μοντέλο να προσαρμόζεται ακόμα και στις μικρές διακυμάνσεις των δεδομένων εκπαίδευσης, οδηγώντας σε έλλειψη γενίκευσης. Τεχνικές για την αποτροπή της υπερπροσαρμογής περιλαμβάνουν την απλοποίηση του μοντέλου μειώνοντας τον αριθμό των παραμέτρων, τη χρήση μεθόδων κανονικοποίησης όπως η L1 ή L2 κανονικοποίηση, και την εφαρμογή διασταυρούμενης επικύρωσης για να διασφαλιστεί ότι η απόδοση του μοντέλου αξιολογείται σε πολλαπλά υποσύνολα των δεδομένων. Μια άλλη αποτελεσματική προσέγγιση είναι η συγκέντρωση περισσότερων δεδομένων εκπαίδευσης, τα οποία μπορούν να βοηθήσουν το μοντέλο να μάθει πιο γενικευμένα μοτίβα.

    \item Κανονικοποίηση και L2 Κανονικοποίηση: Η κανονικοποίηση είναι μια θεμελιώδης τεχνική στη μηχανική μάθηση και τη στατιστική μοντελοποίηση που έχει σχεδιαστεί για να αποτρέπει την υπερεκπαίδευση, η οποία συμβαίνει όταν ένα μοντέλο γίνεται πολύ περίπλοκο και συλλαμβάνει θόρυβο ή τυχαίες διακυμάνσεις στα δεδομένα εκπαίδευσης αντί να γενικεύει καλά σε άγνωστα δεδομένα. Εισάγοντας πρόσθετους περιορισμούς ή ποινές, η κανονικοποίηση βοηθά στη δημιουργία ενός πιο ανθεκτικού μοντέλου που επιτυγχάνει καλύτερα σε νέα δεδομένα.

    Η κανονικοποίηση L2, γνωστή και ως κανονικοποίηση Ridge, είναι μία από τις πιο συχνά χρησιμοποιούμενες μορφές κανονικοποίησης. Λειτουργεί προσθέτοντας μια ποινή αναλογική με το τετράγωνο του μεγέθους των βαρών στη συνάρτηση απώλειας που χρησιμοποιείται κατά την εκπαίδευση. Συγκεκριμένα, αν συμβολίσουμε τα βάρη του μοντέλου ως \( w \), η κανονικοποίηση L2 προσθέτει έναν όρο \(\lambda \sum_{i} w_i^2\) στη συνάρτηση απώλειας, όπου \(\lambda\) είναι μια υπερπαράμετρος που ελέγχει τη δύναμη της κανονικοποίησης. Αυτός ο όρος αποθαρρύνει το μοντέλο από το να δίνει υπερβολική σημασία σε οποιοδήποτε μεμονωμένο χαρακτηριστικό, επιβάλλοντας ποινές σε μεγάλους συντελεστές και ενθαρρύνοντας το μοντέλο να κατανεμηθεί πιο ομοιόμορφα σε όλα τα χαρακτηριστικά.
    
    Το κύριο πλεονέκτημα της κανονικοποίησης L2 είναι ότι βοηθά στην εξομάλυνση της διαδικασίας εκμάθησης και στη μείωση της διακύμανσης του μοντέλου με τη μείωση των βαρών προς το μηδέν, αλλά ποτέ στο μηδέν. Αυτή η επίδραση της μείωσης συχνά οδηγεί σε απλούστερα μοντέλα που είναι λιγότερο ευαίσθητα στις διακυμάνσεις των δεδομένων εκπαίδευσης, γεγονός που ενισχύει τις ικανότητές τους για γενίκευση. Σε αντίθεση με την κανονικοποίηση L1, η οποία μπορεί να οδηγήσει σε σπάνια μοντέλα με κάποιους συντελεστές ακριβώς μηδέν, η κανονικοποίηση L2 έχει την τάση να παράγει μοντέλα όπου όλα τα χαρακτηριστικά συνεισφέρουν σε κάποιο βαθμό, αν και με μικρότερα βάρη. Αυτό το χαρακτηριστικό καθιστά την κανονικοποίηση L2 ιδιαίτερα χρήσιμη σε σενάρια όπου πιστεύουμε ότι όλα τα χαρακτηριστικά έχουν κάποια επίπεδα σημασίας και πρέπει να διατηρηθούν στο μοντέλο.

    \item Ομοσπονδιακή Μάθηση: Η ομοσπονδιακή μάθηση είναι μια τεχνική μηχανικής μάθησης όπου πολλές αποκεντρωμένες συσκευές συνεργάζονται για να εκπαιδεύσουν ένα μοντέλο χωρίς να μοιράζονται τα τοπικά δεδομένα τους. Αντίθετα, κάθε συσκευή εκπαιδεύει το μοντέλο τοπικά και μόνο μοιράζεται τα βάρη του μοντέλου που εκπαίδευσε με έναν κεντρικό διακομιστή. Αυτή η προσέγγιση ενισχύει την ιδιωτικότητα και την ασφάλεια των δεδομένων, ενώ επιπλέον το κεντρικό μοντέλο μπορεί να εκπαιδευτεί από πολλαπλές πηγές. Αντίστοιχα, είναι σημαντικό να σημειωθεί πως μειώνεται πολύ η κίνηση στο δίκτυο, αφού δεν απαιτείται η μεταφορά μεγάλων δεδομένων (π.χ. εικόνων) στον κεντρικό υπολογιστή στον οποίο εκπαιδεύεται το μοντέλο, αλλά γίνεται μόνο μεταφορά βαρών των επιπέδων του νευρωνικού μοντέλου.

\end{enumerate}

\section{Σύγχρονη Έρευνα}

Η Ομοσπονδιακή Μάθηση (Federated Learning - FL) έχει αναδειχθεί ως μία πρωτοπόρο μέθοδο που επιτρέπει τη συνεργατική εκπαίδευση μοντέλων μηχανικής μάθησης, διατηρώντας παράλληλα την ιδιωτικότητα των δεδομένων. Σε αντίθεση με τις παραδοσιακές μεθόδους κεντρικής εκμάθησης που συγκεντρώνουν ακατέργαστα δεδομένα από διανεμημένες πηγές, η Ομοσπονδιακή Μάθηση επιτρέπει σε πολλαπλούς συμμετέχοντες να εκπαιδεύουν κοινά μοντέλα ανταλλάσσοντας παραμέτρους, διασφαλίζοντας έτσι ότι τα ευαίσθητα δεδομένα παραμένουν τοπικά. Αυτή η καινοτόμος προσέγγιση αντιμετωπίζει ζητήματα ιδιωτικότητας, συμμόρφωσης με κανονισμούς όπως ο Γενικός Κανονισμός Προστασίας Δεδομένων (GDPR) και περιορισμούς στη διαμοίραση δεδομένων σε τομείς όπως η υγεία και τα χρηματοοικονομικά.

Πρόσφατες εξελίξεις στην Ομοσπονδιακή Μάθηση επικεντρώνονται στην αντιμετώπιση βασικών προκλήσεων: ετερογένεια, ασφάλεια και δικαιοσύνη. Για να αντιμετωπιστεί η ετερογένεια που χαρακτηρίζει τα περιβάλλοντα Ομοσπονδιακής Μάθησης —η οποία προκύπτει από διαφορετικές κατανομές δεδομένων, ποικίλες αρχιτεκτονικές μοντέλων και άνισες συστημικές δυνατότητες—οι ερευνητές έχουν αναπτύξει νέες τεχνικές, όπως η πολυ-εργασιακή εκμάθηση (multi-task learning), η μεταφορά μάθησης (transfer learning) και η τοπικά συγκεντρωτική μάθηση (clustering-based approaches). Αυτές οι μέθοδοι ενισχύουν τη δυνατότητα της Ομοσπονδιακής Μάθησης να διαχειρίζεται μη ανεξάρτητα και ισοκατανεμημένα δεδομένα (non-IID) και να προσαρμόζει μοντέλα σε ποικίλα περιβάλλοντα κόμβων. Για παράδειγμα, η μετα-εκμάθηση (meta-learning) έχει προσαρμοστεί στις διαδικασίες της Ομοσπονδιακής Μάθησης, επιτρέποντας την εξατομίκευση μέσω βελτιστοποίησης για συγκεκριμένους στόχους κόμβων.

Η ασφάλεια και η ιδιωτικότητα παραμένουν κεντρικές ανησυχίες στην Ομοσπονδιακή Μάθηση, με προσπάθειες που στοχεύουν σε επιθέσεις όπως η αναστροφή βαθμίδων (gradient inversion attacks), οι κακόβουλες παρεμβάσεις (backdoor attacks) και η δηλητηρίαση μοντέλων (model poisoning). Προηγμένες τεχνικές κρυπτογράφησης, όπως η διαφορική ιδιωτικότητα (differential privacy) και η ομομορφική κρυπτογράφηση (homomorphic encryption), ενσωματώνονται όλο και περισσότερο στα πλαίσια της Ομοσπονδιακής Μάθησης για την προστασία ευαίσθητων πληροφοριών κατά τη διαδικασία συγκέντρωσης μοντέλων. Επιπλέον, τα Περιβάλλοντα Αξιόπιστης Εκτέλεσης (Trusted Execution Environments - TEEs) χρησιμοποιούνται για την παροχή ασφάλειας σε επίπεδο υλικού, διασφαλίζοντας ισχυρή άμυνα έναντι πιθανών διαρροών δεδομένων.

Η δικαιοσύνη στην Ομοσπονδιακή Μάθηση έχει επίσης αποκτήσει σημαντική προσοχή, με στόχο την αντιμετώπιση προκαταλήψεων που προκύπτουν από άνισες συνεισφορές πελατών ή άνιση απόδοση μοντέλων μεταξύ συμμετεχόντων. Προσεγγίσεις όπως η ομοσπονδιακή δίκαιη εξομάλυνση (federated fair averaging) και οι επανασταθμισμένες αντικειμενικές συναρτήσεις (reweighted objective functions) στοχεύουν στη διασφάλιση δίκαιων αποτελεσμάτων για όλα τα μέρη, βελτιώνοντας τόσο τη μεμονωμένη όσο και τη συνολική δικαιοσύνη. Αυτές οι μέθοδοι είναι κρίσιμες για την ενίσχυση της εμπιστοσύνης και της συμμετοχής στα συστήματα FL, ιδιαίτερα σε εφαρμογές με ποικίλες βάσεις κόμβων.

Ο τομέας της Ομοσπονδιακής Μάθησης γνωρίζει επίσης την ανάπτυξη κλιμακούμενων και ευέλικτων πλαισίων, όπως το FedLab και το Flower, που απλοποιούν την εφαρμογή της FL σε ετερογενείς συσκευές και περιβάλλοντα. Αυτές οι πλατφόρμες διευκολύνουν πειράματα μεγάλης κλίμακας και γεφυρώνουν το χάσμα μεταξύ έρευνας και πραγματικής εφαρμογής, επιτρέποντας στη FL να εξελιχθεί ως θεμέλιος λίθος της τεχνητής νοημοσύνης με διαφύλαξη ιδιωτικότητας. Στο μέλλον, οι ερευνητές εξετάζουν δυναμικά μοντέλα FL που μπορούν να προσαρμόζονται σε συνεχώς μεταβαλλόμενα περιβάλλοντα δεδομένων, αποκεντρωμένα συστήματα FL για την εξάλειψη της εξάρτησης από κεντρικούς διακομιστές και ενιαία σημεία αναφοράς για την τυποποίηση των αξιολογήσεων απόδοσης μεταξύ των μελετών.

Η Ομοσπονδιακή Μάθηση έχει φέρει επανάσταση σε πολλούς τομείς, επιτρέποντας τη συνεργατική εκπαίδευση μοντέλων ενώ διασφαλίζει την προστασία των ευαίσθητων δεδομένων.

Η Ομοσπονδιακή Μάθηση έχει γίνει ένα κρίσιμο εργαλείο στην υγειονομική περίθαλψη, όπου τα ευαίσθητα δεδομένα ασθενών πρέπει να παραμένουν εμπιστευτικά. Νοσοκομεία και ερευνητικά ιδρύματα χρησιμοποιούν τη FL για τη συνεργατική εκπαίδευση μοντέλων για ιατρική απεικόνιση, διάγνωση ασθενειών και ανακάλυψη φαρμάκων. Για παράδειγμα, επιτρέπει την ανάπτυξη προγνωστικών μοντέλων για την ανίχνευση ασθενειών όπως ο καρκίνος ή οι καρδιαγγειακές παθήσεις, συνδυάζοντας γνώσεις από διανεμημένα σύνολα δεδομένων χωρίς την κοινή χρήση ακατέργαστων ιατρικών αρχείων. Αυτή η συνεργατική προσέγγιση επιταχύνει την καινοτομία διατηρώντας αυστηρά πρότυπα ιδιωτικότητας.

Στον χρηματοοικονομικό τομέα, η Ομοσπονδιακή Μάθηση βελτιώνει την ανίχνευση απάτης, την αξιολόγηση πιστωτικού κινδύνου και τη διαχείριση κινδύνου, αξιοποιώντας διανεμημένα σύνολα δεδομένων από τράπεζες και χρηματοοικονομικούς οργανισμούς. Ευαίσθητα χρηματοοικονομικά δεδομένα, που συχνά περιορίζονται από κανονισμούς και ανησυχίες για την ιδιοκτησία, μπορούν να χρησιμοποιηθούν για την εκπαίδευση ισχυρών προγνωστικών μοντέλων. Για παράδειγμα, τα συστήματα αξιολόγησης πιστοληπτικής ικανότητας που βασίζονται στην Ομοσπονδιακή Μάθηση συγκεντρώνουν πληροφορίες από πολλές τράπεζες, επιτρέποντας δικαιότερες και ακριβέστερες αξιολογήσεις χωρίς να εκθέτουν μεμονωμένα δεδομένα πελατών.

Η Ομοσπονδιακή Μάθηση διαδραματίζει κρίσιμο ρόλο στη βελτίωση εξατομικευμένων υπηρεσιών σε κινητές συσκευές. Εφαρμογές περιλαμβάνουν εξατομικευμένες συστάσεις, προγνωστική πληκτρολόγηση και αναγνώριση φωνής, όπως η χρήση της FL από την Google στο πληκτρολόγιο Gboard για τη βελτίωση της προγνωστικής πληκτρολόγησης χωρίς τη συλλογή δεδομένων κόμβων. Η Ομοσπονδιακή Μάθηση επίσης ενισχύει συσκευές αιχμής σε οικοσυστήματα IoT, επιτρέποντας τη μάθηση σε πραγματικό χρόνο για έξυπνα σπίτια, αυτόνομα οχήματα και φορετές συσκευές, μειώνοντας την κατανάλωση εύρους ζώνης και διασφαλίζοντας την ιδιωτικότητα.

Σε βιομηχανικές ρυθμίσεις, η Ομοσπονδιακή Μάθηση χρησιμοποιείται για τη βελτιστοποίηση των διαδικασιών παραγωγής, την προγνωστική συντήρηση και τη διαχείριση της εφοδιαστικής αλυσίδας. Για παράδειγμα, αισθητήρες σε εργοστάσια μπορούν να εκπαιδεύουν συνεργατικά μοντέλα για την πρόβλεψη βλαβών εξοπλισμού χωρίς να μοιράζονται ευαίσθητα δεδομένα ή δεδομένα ιδιοκτησίας, μειώνοντας τον χρόνο διακοπής λειτουργίας και βελτιώνοντας την αποδοτικότητα.

Η Ομοσπονδιακή Μάθηση υποστηρίζει την ανάπτυξη εφαρμογών για έξυπνες πόλεις, όπως η διαχείριση κυκλοφορίας, η δημόσια ασφάλεια και η βελτιστοποίηση της ενέργειας. Συστήματα παρακολούθησης της κυκλοφορίας, για παράδειγμα, μπορούν να χρησιμοποιήσουν την Ομοσπονδιακή Μάθηση για την εκπαίδευση προγνωστικών μοντέλων από διανεμημένες κάμερες κυκλοφορίας, ώστε να βελτιστοποιήσουν τη ροή της κυκλοφορίας χωρίς την συλλογή ευαίσθητων δεδομένων βίντεο.

Η Ομοσπονδιακή Μάθηση εφαρμόζεται ολοένα και περισσότερο στην εκπαίδευση για προσαρμοστικά συστήματα μάθησης που εξατομικεύουν το περιεχόμενο με βάση την απόδοση των μαθητών. Εκπαιδεύοντας μοντέλα μεταξύ ιδρυμάτων, η Ομοσπονδιακή Μάθηση διευκολύνει τη συνεργατική καινοτομία σε πλατφόρμες ηλεκτρονικής μάθησης, διατηρώντας παράλληλα την εμπιστευτικότητα των δεδομένων των μαθητών.

\section{Διάρθρωση Διπλωματικής Εργασίας}

Στο πλαίσιο της διπλωματικής εργασίας έχουμε ως στόχο την αντιστοίχιση κόμβων με εξυπηρετητές, οι οποίοι ενδιαφέρονται να εκπαιδεύσουν κεντρικά μοντέλα αναγνώρισης και ταξινόμησης εικόνας, ώστε να πετύχουμε το καλύτερο δυνατό στην εκπαίδευση μέσω Ομοσπονδιακής Μάθησης. Στο πρώτο μέρος της διπλωματικής θα μελετήσουμε έναν αλγόριθμο αντιστοίχισης που βασίζεται στη Θεωρία Παιγνίων, ο οποίος αποτελείται από δύο επιμέρους αλγορίθμους που εκτελούν με σειρά Προσεγγιστική και Ακριβή αντιστοίχιση. Στη συνέχεια, θα μελετήσουμε το μοντέλο και την επίδοση της Ομοσπονδιακής Μάθησης για την παραπάνω αντιστοίχιση. Ως δεύτερο μέρος, ακολουθεί η σύγκριση του Αλγορίθμου με Θεωρία Παιγνίων με άλλους αλγορίθμους Μηχανικής Μάθησης και συγκεκριμένα μέσω Ενισχυτικής Μάθησης, όσον αφορά την επίδοση της αντίστοιχισης αλλά και της Ομοσπονδιακής Μάθησης. Στο τρίτο μέρος της διπλωματικής εργασίας θα μελετήσουμε μια διαφορετική προσέγγιση στη λειτουργεία των κόμβων κατά την οποία οι κόμβους θα μπορούν να ελέγξουν τους πόρους τους οποίους διαθέτουν στην Ομοσπονδιακή Μάθηση με στόχο να μεγιστοποιήσουν το κέρδος τους. Στο πλαίσιο αυτό θα δούμε κάποιους αλγορίθμους Μετανοητικής Μάθησης, τους οποίους θα συγκρίνουμε και με τον αρχικό αλγόριθμο Θεωρίας Παιγνίων, ο οποίος δεν μεταβάλλει στους κόμβους τους πόρους που προσφέρουν. Τέλος θα ολοκληρώσουμε την εργασία μελετώντας την σημασία των αποτελεσμάτων μας, αλλά και πιθανές άλλες εφαρμογές του συστήματός που περιγράψαμε.

Στα αποτελέσματα που θα περιγράψουμε, θα μελετήσουμε και θα αναλύσουμε την συμπεριφορά διαφόρων διαμορφώσεων του προβλήματός μας (διαφορετικός αριθμός κόμβων, διαφορετική τεχνική αντιστοίχισης κ.ο.κ.). Για την ορθή εκπόνηση των πειραμάτων, εκτελούμε κάθε ένα από αυτά 5 ή 10 φορές, ανάλογα με την απαιτητικότητά του σε πόρους. Έτσι, πειράματα που αφορούν απλώς αντιστοίχιση των κόμβων με τους εξυπηρετητές τρέχουν 10 φορές για κάθε διαφορετική διαμόρφωση, ενώ πειράματα που αφορούν το χρονοβόρο κομμάτι της Ομοσπονδιακής Μάθησης για την ταξινόμηση των εικόνων, εκτελούνται 5 φορές για κάθε διαφορετική διαμόρφωση.